\begin{figure}[H]
    \centering
    \tikzset{
        nodes={draw=black, fill=tum-lighter-blue, circle}
    }
    \tikzset{
        dots/.style args={#1per #2}{
            line cap=round,
            dash pattern=on 0 off #2/#1
        }
    }

    \begin{subfigure}[t]{0.3\textwidth}
        \centering
        % Figure of a sequence as a graph
        \begin{tikzpicture}
            \node (1) at (1,1) {};
            \node (2) at (2,1) {};
            \node (3) at (3,1) {};
            \node[draw=white, fill=white] at (1,-0.3) {};

            \draw[->] (1) -- (2);
            \draw[->] (2) -- (3);

            \draw[dots = 20 per 1cm,->] (0.5,1) -- (1);
            \draw[dots = 20 per 1cm,->] (3) -- (3.5,1);
        \end{tikzpicture}
        \caption{Sequence}

    \end{subfigure}%
    ~
    \begin{subfigure}[t]{0.3\textwidth}
        \centering
        % Figure of an image as a graph
        \begin{tikzpicture}
            \node (1-1) at (1,1) {};
            \node (1-2) at (1,2) {};
            \node (1-3) at (1,3) {};
            \node (2-1) at (2,1) {};
            \node (2-2) at (2,2) {};
            \node (2-3) at (2,3) {};
            \node (3-1) at (3,1) {};
            \node (3-2) at (3,2) {};
            \node (3-3) at (3,3) {};

            \draw[<->] (1-1) -- (1-2);
            \draw[<->] (1-2) -- (1-3);
            \draw[<->] (2-1) -- (2-2);
            \draw[<->] (2-2) -- (2-3);
            \draw[<->] (3-1) -- (3-2);
            \draw[<->] (3-2) -- (3-3);
            \draw[<->] (1-1) -- (2-1);
            \draw[<->] (2-1) -- (3-1);
            \draw[<->] (1-2) -- (2-2);
            \draw[<->] (2-2) -- (3-2);
            \draw[<->] (1-3) -- (2-3);
            \draw[<->] (2-3) -- (3-3);

            \draw[dots = 20 per 1cm,<->] (1-1) -- (0.5,1);
            \draw[dots = 20 per 1cm,<->] (1-1) -- (1,0.5);
            \draw[dots = 20 per 1cm,<->] (2-1) -- (2,0.5);
            \draw[dots = 20 per 1cm,<->] (3-1) -- (3,0.5);
            \draw[dots = 20 per 1cm,<->] (3-1) -- (3.5,1);
            \draw[dots = 20 per 1cm,<->] (3-2) -- (3.5,2);
            \draw[dots = 20 per 1cm,<->] (3-3) -- (3.5,3);
            \draw[dots = 20 per 1cm,<->] (3-3) -- (3,3.5);
            \draw[dots = 20 per 1cm,<->] (2-3) -- (2,3.5);
            \draw[dots = 20 per 1cm,<->] (1-3) -- (1,3.5);
            \draw[dots = 20 per 1cm,<->] (1-3) -- (0.5,3);
            \draw[dots = 20 per 1cm,<->] (1-2) -- (0.5,2);
        \end{tikzpicture}
        \caption{Image}

    \end{subfigure}%
    ~
    \begin{subfigure}[t]{0.3\textwidth}
        \centering
        % Figure of a general graph
        \begin{tikzpicture}
            \node (A) at (0,0) {};
            \node (B) at (0.2, 1) {};
            \node (C) at (1.2, -0.6) {};
            \node (D) at (-0.7, 1.2) {};
            \node (E) at (-0.5, -0.5) {};
            \node (F) at (-1.2, 0.1) {};
            \node (G) at (1.6, 1.6) {};
            \node (H) at (-0.3, 0.5) {};

            \draw[<->] (A) -- (B);
            \draw[<->] (B) -- (H);
            \draw[<->] (H) -- (A);
            \draw[->] (A) -- (C);
            \draw[->] (E) -- (A);
            \draw[->] (E) -- (H);
            \draw[->] (G) -- (B);
            \draw[->] (G) -- (C);
            \draw[->] (G) -- (C);
            \draw[->] (F) -- (H);
            \draw[->] (F) -- (A);
            \draw[<->] (F) -- (E);
            \draw[->] (D) -- (F);
            \draw[->] (B) -- (C);
            \draw[dots = 20 per 1cm,->] (D) -- (-1,1.6);
            \draw[dots = 20 per 1cm,->] (D) -- (-0.3,1.7);
            \draw[dots = 20 per 1cm,->] (C) -- (1.8,-0.5);
            \draw[dots = 20 per 1cm,<->] (F) -- (-1.8,-0.3);
        \end{tikzpicture}
        \caption{Some arbitrary graph}

    \end{subfigure}
    \caption{Structured data can be represented as a graph in many cases.}
    
\end{figure}