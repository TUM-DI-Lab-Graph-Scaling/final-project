\subsection{Zero Redundancy Optimizer (ZeRO)}
\label{subsection:zero}

The Zero Redundancy Optimizer (ZeRO) is a scaling engine on top of data-parallel training for 
machine learning models. In short, ZeRO removes memory redundancies by partitioning states which 
are used accross all processes such as the optimizer states, gradients and model parameters. 
Since all processes only have to maintain a shard of states, ZeRO should use less memory while 
retaining computational granularity and communication overhead compared to traditional Data 
Parallelism.

Deepspeed contains three levels of ZeRO that progressively partition and distribute more and 
more states of the model. We will go through all stages, explain them and shortly explain our 
integration into OCP.

\begin{figure}[H]
    \centering

    \tikzset{
        gpu/.style={
            shape=rectangle, 
            draw, 
            minimum width=1.8cm,
            minimum height=2.5cm
        },
        data/.style={
            preaction={fill,tum-light-gray},
            shape=rectangle, 
            draw, 
            minimum width=1.8cm, 
            minimum height=0.7cm,
            pattern color=tum-gray,
            pattern=grid
        },
        model/.style={
            preaction={fill,tum-lighter-blue},
            shape=rectangle,
            draw,
            minimum height=0.5cm,
            pattern color=tum-light-blue,
            pattern=grid
        },
        gradient/.style={
            preaction={fill,tum-light-green},
            shape=rectangle,
            draw,
            minimum height=0.5cm,
            pattern color=tum-green,
            pattern=grid
        },
        optimizer/.style={
            preaction={fill,tum-light-orange},
            shape=rectangle,
            draw,
            minimum height=1.5cm,
            pattern color=tum-orange,
            pattern=grid
        },
        empty/.style={
            preaction={fill,tum-lighter-gray},
            shape=rectangle,
            draw,
            minimum width=1.8cm,
            pattern color=tum-light-gray,
            pattern=grid
        }
    }

    \begin{subfigure}[t]{0.5\textwidth}
        \centering
        \begin{tikzpicture}

            \node[gpu, label={\scriptsize GPU $0$}] (0) at (0,0) {};
            \node[gpu, label={\scriptsize GPU $1$}] (1) at (2,0) {};
            \node[gpu, label={\scriptsize GPU $m\!-\!1$}] (n-1) at (4.5,0) {};
            \node[data, label=left:\tiny Data] (0d) at (0,-1.7) {\scriptsize $\mathcal{D}_0$};
            \node[data] (1d) at (2,-1.7) {\scriptsize $\mathcal{D}_1$};
            \node[data] (n-1d) at (4.5,-1.7) {\scriptsize $\mathcal{D}_{m-1}$};
            \draw[dotted, thick] (3,0) -- (3.51,0);
            \draw[dotted, thick] (3,-1.7) -- (3.51,-1.7);
    
            \node[model, minimum width=1.8cm, label=left:\tiny \textcolor{tum-dark-blue}{Model}] (0m) at (0,1) {};
            \node[gradient, minimum width=1.8cm, label=left:\tiny \textcolor{tum-dark-green}{Gradient}] (0g) at (0,0.5) {};
            \node[optimizer, minimum width=1.8cm, label=left:\tiny \textcolor{tum-orange}{Optimizer}] (0o) at (0,-0.5) {};
    
            \node[model, minimum width=1.8cm] (0m) at (2,1) {};
            \node[gradient, minimum width=1.8cm] (0g) at (2,0.5) {};
            \node[optimizer, minimum width=1.8cm] (0o) at (2,-0.5) {};
    
            \node[model, minimum width=1.8cm] (0m) at (4.5,1) {};
            \node[gradient, minimum width=1.8cm] (0g) at (4.5,0.5) {};
            \node[optimizer, minimum width=1.8cm] (0o) at (4.5,-0.5) {};

        \end{tikzpicture}
        \caption*{$\qquad \qquad$\scriptsize \textbf{Stage 0: Standard data parallelism}}
    \end{subfigure}\hspace*{1.4em}%
    ~
    \begin{subfigure}[t]{0.45\textwidth}
        \centering
        \begin{tikzpicture}

            \node[gpu, label={\scriptsize GPU $0$}] (0) at (0,0) {};
            \node[gpu, label={\scriptsize GPU $1$}] (1) at (2,0) {};
            \node[gpu, label={\scriptsize GPU $m\!-\!1$}] (n-1) at (4.5,0) {};
            \node[data] (0d) at (0,-1.7) {\scriptsize $\mathcal{D}_0$};
            \node[data] (1d) at (2,-1.7) {\scriptsize $\mathcal{D}_1$};
            \node[data] (n-1d) at (4.5,-1.7) {\scriptsize $\mathcal{D}_{m-1}$};
            \draw[dotted, thick] (3,0) -- (3.51,0);
            \draw[dotted, thick] (3,-1.7) -- (3.51,-1.7);
    
            \node[model, minimum width=1.8cm] (0m) at (0,1) {};
            \node[gradient, minimum width=1.8cm] (0g) at (0,0.5) {};
            \node[empty, minimum height=1.5cm] (0e) at (0,-0.5) {};
            \node[optimizer, minimum width=0.4cm] (0o) at (-0.7,-0.5) {};
    
            \node[model, minimum width=1.8cm] (0m) at (2,1) {};
            \node[gradient, minimum width=1.8cm] (0g) at (2,0.5) {};
            \node[empty, minimum height=1.5cm] (0e) at (2,-0.5) {};
            \node[optimizer, minimum width=0.4cm] (0o) at (1.7,-0.5) {};
    
            \node[model, minimum width=1.8cm] (0m) at (4.5,1) {};
            \node[gradient, minimum width=1.8cm] (0g) at (4.5,0.5) {};
            \node[empty, minimum height=1.5cm] (0e) at (4.5,-0.5) {};
            \node[optimizer, minimum width=0.4cm] (0o) at (5.2,-0.5) {};

        \end{tikzpicture}
        \caption*{\scriptsize \textbf{ZeRO stage 1}}
    \end{subfigure}

    %\vspace*{0.5em}

    \begin{subfigure}[t]{0.5\textwidth}
        \centering
        \begin{tikzpicture}

            \node[gpu, label={\scriptsize GPU $0$}] (0) at (0,0) {};
            \node[gpu, label={\scriptsize GPU $1$}] (1) at (2,0) {};
            \node[gpu, label={\scriptsize GPU $m\!-\!1$}] (n-1) at (4.5,0) {};
            \node[data, label=left:\tiny Data] (0d) at (0,-1.7) {\scriptsize $\mathcal{D}_0$};
            \node[data] (1d) at (2,-1.7) {\scriptsize $\mathcal{D}_1$};
            \node[data] (n-1d) at (4.5,-1.7) {\scriptsize $\mathcal{D}_{m-1}$};
            \draw[dotted, thick] (3,0) -- (3.51,0);
            \draw[dotted, thick] (3,-1.7) -- (3.51,-1.7);
    
            \node[model, minimum width=1.8cm, label=left:\tiny \textcolor{tum-dark-blue}{Model}] (0m) at (0,1) {};
            \node[empty, minimum height=2cm] (0e) at (0,-0.25) {};
            \node[gradient, minimum width=0.4cm, label=left:\tiny \textcolor{tum-dark-green}{Gradient}] (0g) at (-0.7,0.5) {};
            \node[optimizer, minimum width=0.4cm, label=left:\tiny \textcolor{tum-orange}{Optimizer}] (0o) at (-0.7,-0.5) {};
    
            \node[model, minimum width=1.8cm] (0m) at (2,1) {};
            \node[empty, minimum height=2cm] (0e) at (2,-0.25) {};
            \node[gradient, minimum width=0.4cm] (0g) at (1.7,0.5) {};
            \node[optimizer, minimum width=0.4cm] (0o) at (1.7,-0.5) {};
    
            \node[model, minimum width=1.8cm] (0m) at (4.5,1) {};
            \node[empty, minimum height=2cm] (0e) at (4.5,-0.25) {};
            \node[gradient, minimum width=0.4cm] (0g) at (5.2,0.5) {};
            \node[optimizer, minimum width=0.4cm] (0o) at (5.2,-0.5) {};

        \end{tikzpicture}
        \caption*{$\qquad \qquad$\scriptsize \textbf{ZeRO stage 2}}
    \end{subfigure}\hspace*{1.4em}%
    ~
    \begin{subfigure}[t]{0.45\textwidth}
        \centering
        \begin{tikzpicture}

            \node[gpu, label={\scriptsize GPU $0$}] (0) at (0,0) {};
            \node[gpu, label={\scriptsize GPU $1$}] (1) at (2,0) {};
            \node[gpu, label={\scriptsize GPU $m\!-\!1$}] (n-1) at (4.5,0) {};
            \node[data] (0d) at (0,-1.7) {\scriptsize $\mathcal{D}_0$};
            \node[data] (1d) at (2,-1.7) {\scriptsize $\mathcal{D}_1$};
            \node[data] (n-1d) at (4.5,-1.7) {\scriptsize $\mathcal{D}_{m-1}$};
            \draw[dotted, thick] (3,0) -- (3.51,0);
            \draw[dotted, thick] (3,-1.7) -- (3.51,-1.7);
    
            \node[empty, minimum height=2.5cm] (0e) at (0,0) {};
            \node[model, minimum width=0.4cm] (0m) at (-0.7,1) {};
            \node[gradient, minimum width=0.4cm] (0g) at (-0.7,0.5) {};
            \node[optimizer, minimum width=0.4cm] (0o) at (-0.7,-0.5) {};
    
            \node[empty, minimum height=2.5cm] (0e) at (2,0) {};
            \node[model, minimum width=0.4cm] (0m) at (1.7,1) {};
            \node[gradient, minimum width=0.4cm] (0g) at (1.7,0.5) {};
            \node[optimizer, minimum width=0.4cm] (0o) at (1.7,-0.5) {};
    
            \node[empty, minimum height=2.5cm] (0e) at (4.5,0) {};
            \node[model, minimum width=0.4cm] (0m) at (5.2,1) {};
            \node[gradient, minimum width=0.4cm] (0g) at (5.2,0.5) {};
            \node[optimizer, minimum width=0.4cm] (0o) at (5.2,-0.5) {};

        \end{tikzpicture}
        \caption*{\scriptsize \textbf{ZeRO stage 3}}
    \end{subfigure}
    \vspace*{-0.5em}

    \captionsetup{width=\dimexpr\textwidth-1.5cm\relax}
    \caption{GPU memory utilization with different stages of ZeRO activated.
    The data $\mathcal{D}$ is distributed evenly among the $m$ GPUs into 
    $\mathcal{D}_0, \dots, \mathcal{D}_{m-1}$.}
\end{figure}

\subsubsection{Stage 1 (OS)}

Stage 1 of ZeRO enables Optimizer State Partitioning. Each GPU only holds a shard of optimizer states.
All GPUs do a normal training step over their data subset, however during the optimizer step, each 
GPU only updates and stores optimizer states of its own shard. After the update, all GPUs synchronize 
by an all-gather communication collective to update parameters accross all devices.

Stage 1 required FP16 or Bfloat16 intergation for the OCP codebase. As mentioned in the previous 
section, we encountered overflow problems with FP16, so we mainly used Bfloat16 to enable ZeRO-1.

\subsubsection{Stage 2 (OS+G)}

Stage 2 of ZeRO partitions both optimizer states and gradients. ZeRO-2 generally follows the following 
execution path for each step: Each GPU holds a replica of the entire model (all parameters), however 
only a mutually exlusive portion of the parameters are updated on each of the devices. The GPUs only store 
the shard of gradients and optimizer states, which they need for their particular portion of parameters 
during the forward and backward pass. After doing the optimizer step, the GPUs update each other through 
an \textit{all-gather} NCCL communication collective. 

Stage 2 is very tighly connected to Stage 1 regarding its implementation in Deepspeed. Fortunately 
for us, this meant that after integrating Stage 1, Stage 2 worked out-of-the-box without any problems.

\subsubsection{Stage 3 (OS+G+P)}

Stage 3 of ZeRO partitions optimizer states, gradients and model parameters, thus merging the 
traditional Data Parallelism approach with Model parallelism. Optimizer states and gradients 
are partitioned and synchronized as mentioned in Stage 1 and 2. The model parameter partitioning 
works with the following principles: Each GPU is assigned a shard of parameters. When model 
parameters outside the shard are required for the forward or backward pass, the GPU that owns 
the shard containing the parameters shares them via broadcast. This leads to far more communication 
overhead, however further reduces GPU memory requirements of the overall model.

The integration of ZeRO-3 was fraught with many problems. Gemnet, for example, accesses the 
parameters of the layers directly in the model definition, which leads to problems with 
Deepspeed, since it replaces the tensors with its own wrapper classes that contain parameters 
only latently (parameters that are not in the shard of the GPU must first be loaded via NCCL). 
For this we had to extend Gemnet to first perform an all-gather operation on the parameters, 
so that these are locally available with all GPUs before using them inside the model directly.

Another problem occurred with DimeNet where ZeRO did not automatically partition the model 
parameters. Our solution was to explicitly pass all model parameters as external parameters 
to the Deepspeed engine. This does not provide optimal partitioning of the parameters, but 
at least lets the Zero-3 engine run.

After our integration, the ZeRO optimization can be enabled for all OCP models in the Deepspeed configuration 
file. The specification is as follows:

\begin{json}
"bf16": {
    "enabled": "true"
}
"zero_optimization": {
    "stage": [1|2|3]
}
\end{json}

Note that the current Deepspeed implementation requires bfloat16 floating point precision to be enabled. 
