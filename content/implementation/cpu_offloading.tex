\subsection{Offloading Optimizations}

To further reduce GPU memory usage, Deepspeed contains optimizations to offload 
data and compute from the GPU to CPU or even Non-Volatile Memory express (NVMe) 
memory. 

Offloading optimizations in Deepspeed were introduced in ZeRO-Offload 
\cite*{DBLP:journals/corr/abs-2101-06840}. ZeRO-Offload builds on top of 
the ZeRO Stage 2 optimizer which partitions gradients and optimizer states. 
With offloading enabled, each GPU offloads part of its partition to the CPU. 
The offload engine automatically determines gradients and optimizer states 
which can be computed on the CPU while minimizing communications between CPU 
and GPU and maximize memory savings on the GPU.

To determine the offload strategy, the offload engine represents the model as 
data-flow graph with nodes are model states (parameters, gradients and 
optimiter states) and edges are communication volume between that flows through 
the model during training. ZeRO-Offload uses a two-way partitioning scheme to 
split the graph into CPU and GPU partitionings while trying to optimize CPU 
CPU computation overhead, communication overhead and memory savings. 

Generally speaking, offloading is an scale-up approach to utilize as much 
resources on each machine as possible, including CPU cores and memory. 
Offloading should reduce GPU memory consuption at the cost of CPU memory and 
core utilization, while the overall runtime increases.

ZeRO-Offload for Zero Stage 2 is enabled by specifing the offload optimizer 
configuration inside of the ZeRO configuration:

\begin{json}
{
    "zero_optimization": {
        "stage": 2,
        "offload_optimizer": {
            "device": "[cpu|nvme]"
        }
    }
}
\end{json}

ZeRO-Infinity \cite*{DBLP:journals/corr/abs-2104-07857} further features of 
ZeRO-Offload by building on top of ZeRO-3 (Stage 3) and also introducing NVMe 
offloading.

ZeRO-Infinity builds on top of the offload engine of ZeRO-Offload by also 
mapping the model to a data-flow graph as described above. However, the 
novel infinity offload engine also allows assigning data-flow partitions 
to NVMe memory and extends the graph by including activation memory hence 
also allowing offloading activiation checkpoints to CPU memory or NVMe. 
ZeRO-Infinity also introduces the concept of memory-centric tiling for 
working memory to split larger layers in the model into smaller tiles, 
allowing the offload engine to further improve the partitioning scheme and 
prevent GPU memory overflows with single large layers which often occured 
before in large language models.   To sum it up, ZeRO-Infinity allows 
models that are significantly larger than the GPU memory to be trained as 
long as enough CPU or NVMe is available. 

Similiar to ZeRO-Offload, after our integration of ZeRO-3 into the OCP 
code base, infinity offloading worked out-of-the-box. ZeRO-Infinity's
offload engine can be enabled in Stage 3 using the two ZeRO configurations 
\textit{offload\_optimizer} and \textit{offload\_param} to offload 
either optimizer states or entire parts of the model:

\begin{json}
{
    "zero_optimization": {
        "stage": 3,
        "offload_optimizer": {
            "device": "[cpu|nvme]"
        },
        "offload_param": {
            "device": "[cpu|nvme]"
        }
    }
}
\end{json}