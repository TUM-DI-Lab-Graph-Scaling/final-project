\subsection{Evaluation Setup}

We evaluated all models on a single machine with an AMD Epyc 7542 (128 CPU-Cores), 528 GB memory, 
and 8 Nvidia RTX A6000 with 48 GB GDDR6-memory. The GPUs were not connected by NVLink, so we used 
P2P communication through PCI. Practically speaking, this means that we set \texttt{NCCL\_P2P\_LEVEL} to 
"PIX".

\begin{enumerate}[align=left]
    \item[\textbf{GemNet}] The first model we evaluated with our DeepSpeed enhancements was 
    the force-centric GemNet-dT scaled up to roughly 292 million parameters corresponding to 
    Gemnet-XL \cite*{https://doi.org/10.48550/arxiv.2203.09697}. 
    The model consists of 6 interactions blocks, an atom embedding size of 128, an edge 
    embedding size of 1536, and triplet embedding size of 384. All other hyperparameters are equal 
    to Gemnet-dT as it was configured for the OCP benchmark evaluation.

    \item[\textbf{DimeNet}] Furthermore we evaluated the energy-centric DimeNet++ with the configuration 
    defined in \cite*{https://doi.org/10.48550/arxiv.2203.09697}. The model has 4 interaction blocks, a hidden 
    dimension of 2048, an output block of 1536, and triplet dimension of 256, amounting to approximately 
    240 million parameters. Again, all other hyperparameters are equal to DimeNet++ as it was configured 
    for the OCP benchmark evaluation.
\end{enumerate}

For detailed analysis we gathered metrices from all our experiment runs regarding epoch runtimes, 
CPU memory usage, GPU memory usage, as well as GPU memory reserved or allocated by PyTorch directly.
We collected the data with a custom profling framework for PyTorch which tracks each individual phase 
of computations, in specific the dataloading, forward pass and backward pass, to further investigate 
internal behaviour of our applied optimizations. 
In addition we used integrated the PyTorch profiler\footnote{\url{https://pytorch.org/docs/stable/profiler.html}}
to collect step-by-step performance metrics, including information about communcation collectives and 
volume or breakdown of calls to CUDA operations. PyTorch profiler also includes plugins to 
load and visualize the profiling data in Tensorboard, which allowed us to do ad-hoc profiling 
even when the experiment was still running.



