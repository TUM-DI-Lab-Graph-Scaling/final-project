\section{Conclusion}
\label{section:conclusion}

In this work, we scaled force and energy predictions of organic molecules on
potential catalyst surfaces for a wide variety of recently proposed GNN architectures
in the domain of atomic simulations. This was achieved by integrating Microsoft DeepSpeed 
into the OCP project, facilitating the exploration and use of DeepSpeed optimizations
for the GNN models that are part of the OCP repository. This does not
only include GemNet and DimeNet++, but also---as of recently---the current state-of-the-art
GemNet-OC \cite{https://doi.org/10.48550/arxiv.2204.02782}.
Additionally, we developed a module for OCP that enables comprehensive tracking
of memory- and runtime-related metrics. 

We evaluated memory and runtime savings on scaled-up versions of GemNet-dT
($\sim$292M parameters) and DimeNet++ ($\sim$216M parameters) which correspond to
GemNet-XL and DimeNet++-XL \cite{https://doi.org/10.48550/arxiv.2203.09697}, 
the state-of-the-art on the OCP leaderboard when our project phase began. 

For our GemNet setup, we observed that using DeepSpeed's ZeRO optimization 
can significantly reduce memory footprints during training, resulting in 
almost 50\% less reserved memory or 87\% less allocated memory on the S2EF task
(depending on the configuration of DeepSpeed) compared to standard data parallel training. 
In terms of runtime, however, in almost all of our evaluations we were not able to speed up 
training as communication overhead increased when 
nontrivial DeepSpeed features were activated. Runtime-wise, just training in half
precision has mostly yielded the best results.

In our GemNet evaluation, stage 3 of the ZeRO optimizer which corresponds to full model parallelism on top
of data parallelism did not outperform the simpler stages 1 and 2 neither with respect to
memory nor runtime. We suspect that for stage 3, being originally designed for trillion-parameter
transformer models that cannot even be stored on single GPUs, our models
were simply too small to see any memory savings.

In contrast, our results for DimeNet++ were rather unexpected as communication
between the processes got very slow when we activated stage 1 and 2 of ZeRO.
We ruled out various potential problems related to our DeepSpeed configuration and
the model itself (like the energy-centric DimeNet++ calculating second order gradients 
in the backward pass) and suspect that this might be a problem related to the NCCL communication on our evaluation 
machine.

Since each configuration was only trained for one epoch and not all of the OC20 data was 
used for training due to time constraints, it would be interesting to see in the future if our results 
also prove true for longer and more scaled-up training and if the activation of the DeepSpeed features 
influences performance of the models in some way. 
In addition, it could also be very interesting to explore if the results we obtained for our GemNet setup
can be transferred to the current state-of-the-art architecture, GemNet-OC, which does not only
consider 3- but also 4-way interactions.
Apart from all of this, we would be curious to see some of the DeepSpeed features being
combined with other parallization strategies in the domain of atomic simulations, like 
graph parallelism \cite{https://doi.org/10.48550/arxiv.2203.09697} which was also added to
OCP recently during our project phase.

