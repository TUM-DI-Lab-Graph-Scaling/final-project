\section{Related Work}
\label{section:related_work}

Scaling of machine learning models has a rich scientific research history, mainly since the field 
represents an interesting and active collaboration between high performance computing, distributed 
systems, and deep learning communities. So in this section we will highlight literature in the area 
of accelerating deep learning that are similar to our project or complement our work.

A. Sriram et al. \cite*{https://doi.org/10.48550/arxiv.2203.09697} from Meta FAIR propose a very 
different approach to scaling Message Parsing Graph Neural Networks for atomic simulations, wherein 
the automata structures are internally partitioned across multiple GPUs. In their approach, known 
as graph parallelism, the higher-order interactions, represented in the graph as triplets 
(two connected edges with associated angles of the connected atoms), are partitioned so that 
each GPU receives a subset of the triplet and updates and aggregates them locally. The 
triplet messages are then gathered on all GPUs, allowing subsequent edge and node-level layers 
to be computed as before. Using Graph Parallelism, the authors scaled up Dimenet++ and Gemnet, 
referring to their large models as Dimenet++-XL and Gemnet-XL. Gemnet-XL, containing close to 300 
million model parameters, is the largest known model fully trained on OC20 for the OCP benchmark 
to date, and was also at the top of the leaderboard for both IS2RE and S2EF until the release of 
Gemnet-OC. 

For more general porpose Graph Neural Network architectures like Graph Attention Networks 
\cite*{10.48550/ARXIV.1710.10903} or GraphSAGE \cite*{10.48550/ARXIV.1706.02216}, sampling 
is mostly used to avoid neighborhood explosions when building the Message Flow Graphs. To also 
allow training on large-scale graphs, both popular Geometric Deep Learning frameworks PyTorch 
Geomtric (PyG) \cite*{Fey2019} and Deep Graph Library (DGL) \cite*{10.48550/ARXIV.1909.01315} 
offer support for scale-out strategies. For PyG, there is Torch Quiver
\footnote{\url{https://github.com/quiver-team/torch-quiver}}, which optimizes and scales 
single-node multi-GPU training. In DGL, multi-GPU training and optimizations are directly 
integrated into the codebase. In first iterations of our project, we evaluated both Quiver 
on Pytorch Geometric and DGL. In short, Torch Quiver is unfortunately very poorly maintained 
and currently not useable from our point of view. DGL offers great support for multi-GPU and 
our tests were very positive, however most atomic simulations models are implemented on top 
of Pytorch Geomtric. Consequently, we decided against DGL because the effort to reimplement 
Dimenet, Gemnet, and Gemnet-OC in DGL would have been too costly. 


