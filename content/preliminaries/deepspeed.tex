\subsection{Microsoft DeepSpeed}

Microsoft DeepSpeed\footnote{\url{https://github.com/microsoft/DeepSpeed}} is
a deep learning training optimization library built on top of PyTorch enabling
large-scale distributed neural network training and making training on multiple devices 
(or machines) more memory- and time-efficient. Particularly for large-scale training of
transformer models like Megatron-LM \cite{10.48550/ARXIV.1909.08053}, DeepSpeed has
already been used extensively and proven very 
useful\footnote{\url{https://github.com/microsoft/Megatron-DeepSpeed}}.
As we will dive deeper into DeepSpeed's optimizations and their integration 
into the OCP codebase in the following chapter~\ref{section:implementation}, this 
section is merely intended as a high-level overview of the optimizations that 
DeepSpeed offers.

At the heart of DeepSpeed lies the \textit{Zero Redundancy Optimizer} 
\cite{https://doi.org/10.48550/arxiv.1910.02054,DBLP:journals/corr/abs-2101-06840,DBLP:journals/corr/abs-2104-07857},
or short \textit{ZeRO}, which is a sophisticated and broad parallelization strategy 
combining data, model and pipeline parallelism paradigms.
As it was already implied in the previous section~\ref{subsubsec:data-model-pipeline},
both model and data parallelism separately have their own flaws: While data parallelism
is memory inefficient as there is a lot of redundancy in storing a copy of the whole 
model in each data-parallel process, model parallelism can impose significant communication
overheads. 

At this point, let us just briefly remind ourselves of the types of memory consumption 
that occur when training neural networks: The main components that consume memory are the 
model itself (including parameters, gradients and optimizer states), 
the current data batch, as well as residual state memory, namely activations that are 
computed during the forward pass or temporary buffers.

ZeRO, as introduced initially in \cite{https://doi.org/10.48550/arxiv.1910.02054},
optimizes both model memory consumption and residual memory consumption in the
following ways:

\begin{itemize}
    \item \textit{ZeRO-DP} (\enquote{data parallelism}) is based on standard
          data parallelism and is designed for large models with significant
          memory footprints. ZeRO-DP can be seen as a combination of data and model
          parallelism in which (parts of) the model states are partitioned
          among the data-parallel processes, while a dynamic communication
          schedule ensures relatively low communication overhead. Various stages
          determining the scope of the model state partitioning can be progressively
          activated, like stage 1 (partitioning of optimizer states among the processes), 
          stage 2 (partitioning of optimizer states and gradients) as well as 
          the final stage 3 (partitioning of optimizer states, gradients, and 
          parameters) which corresponds to full model parallelism.
    \item \textit{ZeRO-R} (\enquote{residual memory}) tackles residual memory consumption: 
          Generally, activation memory consumption can be decreased by 
          \textit{activation checkpointing}, which means that certain activations
          are released from memory after computation in the forward pass and
          recomputed when needed in the backward pass---at the cost of this additional
          computation time. Inter alia, ZeRO-R now also partitions such activation 
          checkpoints across the data-parallel processes and gathers them again when 
          needed; apart from that, activations might also be moved to CPU memory. 
\end{itemize}

\textit{ZeRO-Offload} \cite{DBLP:journals/corr/abs-2101-06840} and 
\textit{ZeRO-Infinity} \cite{DBLP:journals/corr/abs-2104-07857} are feature extensions of ZeRO
that were added at a later point in time to enable offloading of model states to CPU memory 
for very large models in order to further increase GPU memory savings. While ZeRO-Offload 
enables offloading of optimizer states in stage 2 of ZeRO-DP, ZeRO-Infinity
enables both offloading of model parameters and optimizer states in stage 3.