\subsection{Open Catalyst Project}

Catalysis plays a crucial role in the chemical industry, including energy storage and conversion in 
fuel cells or the production of armonia for fertilizers, by enabling new reactions and improved 
process efficiencies \cite{Chanussot_2021}. Problematic is that the number of materials that can 
be used or modified in catalysis is very large and modeling of materials during reaction is very 
complex and compute intensive.

Machine Learning methods and in particular Graph Neural Networks have shown great success in 
molecular simulations in recent years. Beeing able to effeciently and accurately predict the 
forces and energy of inorganic and organic interfaces for use in catalysis avoids the 
computational bottlenecks of traditional simulation tools, such as Density Functional Theory 
(DFT) \cite{doi:10.1021/ed5004788}. 

The Open Catalyst Project \cite*{Chanussot_2021} was launched in 2020 by Facebook AI and 
aims to provide a unified dataset and baseline models for predicting Forces and Energy in molecular 
simulations of catalysts. The project provides the Open Catalyst 2020 (OC20) dataset, which contains 
1.2 million relaxations of molecular adsorptions onto surfaces simulated with DFT. Fundamentally this 
precomputed dataset opens the door for precise predictions through ML models and allows large-scale 
explorations of new catalysts.

Recently the OCP team released a new version version of the dataset, Open Catalyst 2022 (OC22) 
\cite{https://doi.org/10.48550/arxiv.2206.08917}