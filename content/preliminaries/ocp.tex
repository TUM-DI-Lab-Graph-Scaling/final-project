\subsection{Open Catalyst Project}

Catalysis plays a crucial role in the chemical industry, including energy storage and conversion in 
fuel cells or the production of ammonia for fertilizers by enabling new reactions and improved 
process efficiencies \cite{Chanussot_2021}. Problematic is that the number of materials that can 
be used or modified in catalysis is very large and modeling of materials during reactions is very 
complex and compute-intensive.

In recent years, machine learning methods and in particular GNNs have shown great success in 
catalyst discovery: Being able to efficiently and accurately predict the 
forces and energy of inorganic and organic interfaces for use in catalysis avoids the 
computational bottlenecks of traditional simulation tools (see \ref{subsubsec:atomic-simulations-general}). 

The Open Catalyst Project\footnote{\url{https://github.com/Open-Catalyst-Project/ocp}} \cite*{Chanussot_2021} 
is a joint effort by Facebook AI and Carnegie Mellon University's Department of Chemical Engineering which was 
launched in 2020 and aims to provide a unified dataset and baseline models for predicting forces and energy in 
molecular simulations of catalysts. The project provides the Open Catalyst 2020 (OC20) dataset, which contains 
about 1.2 million relaxations of molecular adsorptions onto surfaces simulated with DFT. Fundamentally this 
precomputed dataset opens the door for precise predictions through machine learning models and allows
for large-scale explorations of new catalysts.

Furthermore, the Open Catalyst Project offers unified implementations of DimeNet \ref{subsubsec:dimenet} and 
GemNet \ref{subsubsec:gemnet}, as well as its subvariants. Developers can clone the repository and select 
an model to be trained or evaluated on force-centric or energy-centric tasks with precomputed simulations of 
catalysts. New models can be submitted to the evaluation server and get listed with their metrics on the OCP 
leaderboard\footnote{\url{https://opencatalystproject.org/leaderboard.html}}.

Recently, the OCP team released a new version version of the dataset, Open Catalyst 2022 (OC22) 
\cite{https://doi.org/10.48550/arxiv.2206.08917}.