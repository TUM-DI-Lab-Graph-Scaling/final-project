\section{Preliminaries}

\subsection{Graph Neural Networks}

In most of the contemporary supervised machine learning tasks, we are usually given 
datasets
\[ 
    \mathcal{D} \: \coloneqq \: 
    \set{(\xx_i, \yy_i)}_{i=1}^n 
    \quad \text{with} \quad \xx_i \in \R^{d}, \: \yy_i \in \R^k 
\]
where the inputs $\xx_i$ are assumed to be independently and identically 
distributed and we want to generally predict the labels $\yy$ from the $\xx$ using
some parametric function
\[
    \fct{f_\theta}{\R^d}{\R^k}, \quad
    \hat{\yy} \: \coloneqq \: f_\theta(\xx),
\]
i.e. we want to find some parameter $\theta$ in a parameter space $\Theta$ such that
the predictions $\hat{\yy}$ are somewhat close to the targets $\yy$ which is usually
done by minimizing a loss function (sometimes modeled as negative log-likelihood)
\[
    L(\mathcal{D}, \theta) \: \coloneqq \: \sum_{i=1}^n l(f_\theta(\xx_i), \yy_i)
\]
over the parameter space. Over the last decades, neural networks have proven to work 
really well for implementing these functions $f_\theta$ and are usually trained by 
variants of gradient descent.

However, for many complex structured non-Euclidean types of data, this approach might 
not be the best fit. For example, just consider the two following settings:
\begin{itemize}
    \item Suppose we are given molecule structures as input graphs $\xx_i$ and, as an 
          outcome, we want to predict the energy $\yy_i$ of the respective structures. 
          Clearly, there is no straightforward way to embed a graph into some $\R^d$.
    \item Suppose our inputs $\xx_i$ are certain features of users in a social network
          and we want to predict interaction $\yy_i$ with some newly uploaded content. In this 
          case, the iid assumption of the individual data points $(\xx_i, \yy_i)$ is 
          completely unreasonable because users themselves might share common interest 
          based on their interactions---or even interact with each other to make each 
          other aware of the content. Therefore it would be sensible to incorporate the 
          existing connections between the data points somehow---so essentially one would 
          like models which operate on one giant input graph with interconnected and 
          dependent data points as nodes.
\end{itemize}
Surely, one could find some representation of the above input data and try to use some
conventional method (i.e. try to let a neural network learn these structures on its own),
however this would not only be inefficient but also completely unfeasible for many large 
modern datasets. 

For simpler structured data like sequences or images, neural
network architectures that exploit these structures have existed for quite some time, like 
Recurrent Neural Networks (RNNs) and
Convolutional Neural Networks (CNNs) respectively. Particularly the introduction of CNNs 
in the early 2010s marked a huge step forward for tasks like image classification 
compared to dense fully-connected neural networks: CNNs leverage the spatial structure of
images as rectangular grids by applying convolutional filters like sliding windows on small
rectangular parts of the input image. 

Even though the topological structure of sequences and images is quite simple and regular
compared to some general graph, we can still see both of them as graphs themselves. 

% Structured data figure that shows sequences and images as graphs.
\begin{wrapfigure}{r}{0.3\textwidth}
    \centering
    \tikzset{
        nodes={
            draw=black, 
            fill=tum-lighter-blue, 
            circle,
            inner sep=3
        }
    }
    \tikzset{
        dots/.style args={#1per #2}{
            line cap=round,
            dash pattern=on 0 off #2/#1
        }
    }

    \begin{subfigure}[t]{0.2\textwidth}
        \centering
        % Figure of a sequence as a graph
        \begin{tikzpicture}
            \node (1) at (1,1) {};
            \node (2) at (2,1) {};
            \node (3) at (3,1) {};

            \draw[->] (1) -- (2);
            \draw[->] (2) -- (3);

            \draw[dots = 20 per 1cm,->] (0.5,1) -- (1);
            \draw[dots = 20 per 1cm,->] (3) -- (3.5,1);
        \end{tikzpicture}
        \caption*{\scriptsize Sequence}

    \end{subfigure}

    \vspace*{0.5em}

    \begin{subfigure}[t]{0.2\textwidth}
        \centering
        % Figure of an image as a graph
        \begin{tikzpicture}
            \node (1-1) at (1,1) {};
            \node (1-2) at (1,2) {};
            \node (1-3) at (1,3) {};
            \node (2-1) at (2,1) {};
            \node (2-2) at (2,2) {};
            \node (2-3) at (2,3) {};
            \node (3-1) at (3,1) {};
            \node (3-2) at (3,2) {};
            \node (3-3) at (3,3) {};

            \draw[<->] (1-1) -- (1-2);
            \draw[<->] (1-2) -- (1-3);
            \draw[<->] (2-1) -- (2-2);
            \draw[<->] (2-2) -- (2-3);
            \draw[<->] (3-1) -- (3-2);
            \draw[<->] (3-2) -- (3-3);
            \draw[<->] (1-1) -- (2-1);
            \draw[<->] (2-1) -- (3-1);
            \draw[<->] (1-2) -- (2-2);
            \draw[<->] (2-2) -- (3-2);
            \draw[<->] (1-3) -- (2-3);
            \draw[<->] (2-3) -- (3-3);

            \draw[dots = 20 per 1cm,<->] (1-1) -- (0.5,1);
            \draw[dots = 20 per 1cm,<->] (1-1) -- (1,0.5);
            \draw[dots = 20 per 1cm,<->] (2-1) -- (2,0.5);
            \draw[dots = 20 per 1cm,<->] (3-1) -- (3,0.5);
            \draw[dots = 20 per 1cm,<->] (3-1) -- (3.5,1);
            \draw[dots = 20 per 1cm,<->] (3-2) -- (3.5,2);
            \draw[dots = 20 per 1cm,<->] (3-3) -- (3.5,3);
            \draw[dots = 20 per 1cm,<->] (3-3) -- (3,3.5);
            \draw[dots = 20 per 1cm,<->] (2-3) -- (2,3.5);
            \draw[dots = 20 per 1cm,<->] (1-3) -- (1,3.5);
            \draw[dots = 20 per 1cm,<->] (1-3) -- (0.5,3);
            \draw[dots = 20 per 1cm,<->] (1-2) -- (0.5,2);
        \end{tikzpicture}
        \caption*{\scriptsize Image}

    \end{subfigure}

    \vspace*{0.5em}

    % \begin{subfigure}[t]{0.25\textwidth}
    %     \centering
    %     % Figure of a general graph
    %     \begin{tikzpicture}
    %         \node (A) at (0,0) {};
    %         \node (B) at (0.2, 1) {};
    %         \node (C) at (1.2, -0.6) {};
    %         \node (D) at (-0.7, 1.2) {};
    %         \node (E) at (-0.5, -0.5) {};
    %         \node (F) at (-1.2, 0.1) {};
    %         \node (G) at (1.6, 1.6) {};
    %         \node (H) at (-0.3, 0.5) {};

    %         \draw[<->] (A) -- (B);
    %         \draw[<->] (B) -- (H);
    %         \draw[<->] (H) -- (A);
    %         \draw[->] (A) -- (C);
    %         \draw[->] (E) -- (A);
    %         \draw[->] (E) -- (H);
    %         \draw[->] (G) -- (B);
    %         \draw[->] (G) -- (C);
    %         \draw[->] (G) -- (C);
    %         \draw[->] (F) -- (H);
    %         \draw[->] (F) -- (A);
    %         \draw[<->] (F) -- (E);
    %         \draw[->] (D) -- (F);
    %         \draw[->] (B) -- (C);
    %         \draw[dots = 20 per 1cm,->] (D) -- (-1,1.6);
    %         \draw[dots = 20 per 1cm,->] (D) -- (-0.3,1.7);
    %         \draw[dots = 20 per 1cm,->] (C) -- (1.8,-0.5);
    %         \draw[dots = 20 per 1cm,<->] (F) -- (-1.8,-0.3);
    %     \end{tikzpicture}
    %     \caption*{\scriptsize Arbitrary graph}
    %
    %\end{subfigure}
    \caption{Structured data as graphs.}
    
\end{wrapfigure}

Graph Neural 
Networks (GNNs) can now be regarded as a generalization of CNNs and RNNs to general graph structures.
More precisely, we can say that GNNs exploit the structure of the input graph(s) by incorporating very
reasonable assumptions about its predicting functions like equivariance (in the case of node prediction 
tasks) or invariance (for example in the case of global predictions) with respect to permutations of the 
input nodes. Indeed, both of the above introductory examples in which conventional methods are not applicable
can be modeled with GNNs.

We introduce GNNs via the Graph Network/Extended Graph Network (GN/EGN) framework, which defines
a common language for a quite general class of graph-input functions made out of 
individual building blocks. These GNs then subsume GNNs---and GNNs can essentially just be 
considered GNs whose individual building blocks are neural networks. 
EGNs, which we will introduce afterwards, are even more expressive than normal 
GNs as they also model higher-order interactions between graph vertices (i.e. interactions 
between 3 or more vertices, or---alternatively---between two graph edges). Especially
recent GNNs for atomic simulations that make use of such higher-order interactions, as the ones 
we will describe in subsection~\ref{subsec:atomic-simulations}, can be expressed elegantly in the
language of the EGN framework.

Within the GN framework, a graph is modeled as a 3-tuple
\[ G \: = \: (\uu, V, E) \text{,} \]
where $\uu \in \R^{d_u}$ is a global attribute of the graph. $V$ and $E$ are the set of nodes and
set of edges respectively,
\begin{align*}
    V \: &= \: \set{\vv_i}_{i=1}^{n_v},             &\quad \vv_i \in \R^{d_v}&; \\
    E \: &= \: \set{(\ee_k, r_k, s_k)}_{k=1}^{n_e}, &\quad \ee_k \in \R^{d_e}&, \; r_k, s_k \in \set{1, \dots, n_v};
\end{align*}
where $\vv_i$ represents the node attributes, and $\ee_k$ is the attribute of an edge
going from $s_k$ to $r_k$.

A GN block now consists of the update functions $\phi_e$, $\phi_v$ and $\phi_u$ for edges,
nodes and attributes respectively, as well as aggregation functions $\rho_{e \to v}$,
$\rho_{e \to u}$, and $\rho_{v \to u}$, with
\begin{align*}
    \ee_k' \: &\coloneqq \: \phi_e(\ee_k, \vv_{r_k}, \vv_{s_k}, \uu), 
    &\overline{\ee}'_i \: &\coloneqq \rho_{e \to v}(E_i'), \\
    \vv_i' \: &\coloneqq \: \phi_v(\overline{\ee}'_i, \vv_i, \uu), 
    &\overline{\ee}' \: &\coloneqq \: \rho_{e \to u}(E'), \\
    \uu' \: &\coloneqq \: \phi_u(\overline{\ee}', \overline{\vv}', \uu),
    &\overline{\vv}' \: &\coloneqq \: \rho_{v \to u}(V');
\end{align*}
where $E_i' \coloneqq \setcomp{(\ee_k', r_k, s_k)}{k = 1, \dots, n_e, \; r_k = i}$
is the set of incoming edges at node $i$, $E' \coloneqq \set{(\ee_k', r_k, s_k)}_{k=1}^{n_e}$
contains all updated edges and $V' \coloneqq \set{\vv_i'}_{i=1}^{n_v}$ consists
of the updated nodes of the graph. All in all, this results in the updated graph 
$G' = (\uu', V', E')$ being the output of the GN block.
The computations performed by a GN block can also be described in the following phases:

\begin{enumerate}[align=left]
    \item[\textbf{Edge Update:}] The edges are updated individually based on their respective edge attribute, 
    their source and receiver node attributes, and the graph's global attribute using the
    update function $\phi_e$.
    \item[\textbf{Edge Aggregation:}] The new edge attributes are aggregated on each node
    as well as the whole graph using the aggregation functions $\phi_{e \to v}$ and
    $\phi_{e \to u}$ respectively.
    \item[\textbf{Node Update}:] The nodes are updated individually based on their respective
    node attributes, the newly aggregated messages from the incoming edges, and the graph's
    global attribute using the update function $\phi_v$.
    \item[\textbf{Node Aggregation:}] The new node attributes are aggregated on the whole
    graph via the aggregation function $\phi_{v \to u}$.
    \item[\textbf{Global Update:}] The graph's global attribute is updated based on its
    predecessor and the newly aggregated edge and node attributes.
\end{enumerate}

Furthermore, the aggregation functions have to be invariant w.r.t. permutations of the
input node or edge attributes, i.e. do not depend reordering of the inputs. In many cases,
these aggregation functions are kept simple and are just a sum or mean operation over
the inputs.

Now, a GN is just a sequence of such GN blocks which successively updates the input graph's
attributes.

% Figure that demonstrates the five update phases of an GN block.
\begin{figure}[H]
    \centering
    \tikzset{
        n_inactive/.style={draw=tum-gray, fill=white, text=tum-gray, circle},
        n_active/.style={draw=tum-gray, fill=tum-lighter-blue, text=tum-dark-blue, circle},
        n_updated/.style={draw=tum-gray, fill=tum-light-orange, text=tum-orange, circle},
        e_inactive/.style={draw=tum-gray},
        e_active/.style={draw=tum-blue, thick},
        e_updated/.style={draw=tum-orange, thick}
    }

    \makeatletter
    \tikzset{circle split part fill/.style  args={#1,#2}{%
    alias=tmp@name,
    postaction={%
        insert path={
        \pgfextra{% 
        \pgfpointdiff{\pgfpointanchor{\pgf@node@name}{center}}%
                    {\pgfpointanchor{\pgf@node@name}{east}}%            
        \pgfmathsetmacro\insiderad{\pgf@x}
        \fill[#1] (\pgf@node@name.base) ([xshift=-\pgflinewidth]\pgf@node@name.east) arc
                            (0:180:\insiderad-\pgflinewidth)--cycle;
        \fill[#2] (\pgf@node@name.base) ([xshift=\pgflinewidth]\pgf@node@name.west)  arc
                            (180:360:\insiderad-\pgflinewidth)--cycle;   
            }}}}}  
 \makeatother

    \begin{subfigure}[t]{0.18\textwidth}
        \centering
        \begin{tikzpicture}
            % Edge Update
            \draw[draw=white] (-0.6,-3.25) rectangle (2,3.25) {};
            \draw[rounded corners] (-0.6,-1.5) rectangle (2,1.5) {};
            \node[text=tum-dark-blue] (T) at (1.65,1.2) {\scriptsize $\uu$};
    
            \node[n_active, inner sep=1] (A) at (1,0) {\scriptsize $\vv_{r_k}$};
            \node[n_active, inner sep=1] (B) at (0,0.6) {\scriptsize $\vv_{s_k}$};
            \node[n_inactive] (C) at (0.3,-0.7) {};
            \node[n_inactive] (D) at (1.4, -1) {};
    
            \draw[->, e_updated] (B) -- (A) node[text=tum-orange, near end, above] {\scriptsize $\ee_k'$};
            \draw[->, e_inactive] (C) -- (A);
            \draw[->, e_inactive] (D) -- (A);
            \draw[<->, e_inactive] (C) -- (D);
        \end{tikzpicture}
        \vspace*{0.5em}
        \caption*{\footnotesize Edge Update}
    \end{subfigure}%
    ~
    \begin{subfigure}[t]{0.18\textwidth}
        \centering
        \begin{tikzpicture}
            %Edge Aggregation
            \draw[rounded corners] (-0.6,-1.5) rectangle (2,1.5) {};
            \node[text=tum-gray] (T) at (1.65,1.2) {\scriptsize $\uu$};
    
            \node[n_inactive, inner sep=1, circle split, circle split part fill={white,tum-light-orange}] (A) at (1,0) {\scriptsize $\vv_i$ \nodepart{lower} \textcolor{tum-orange}{\scriptsize $\overline{\ee}_i'$}};
            \node[n_inactive] (B) at (0,0.6) {};
            \node[n_inactive] (C) at (0.3,-0.7) {};
            \node[n_inactive] (D) at (1.4, -1) {};
    
            \draw[->, e_active] (B) -- (A);
            \draw[->, e_active] (C) -- (A);
            \draw[->, e_active] (D) -- (A);
            \draw[<->, e_inactive] (C) -- (D);

            \draw[rounded corners] (-0.6,-5) rectangle (2,-2) {};
            \node[text=tum-gray] (T) at (1.65,-2.3) {\scriptsize $\uu$};
            \node[text=tum-orange] (T) at (1.7,-2.6) {\scriptsize $\overline{\ee}'$};
    
            \node[n_inactive] (A) at (1,-3.5) {};
            \node[n_inactive] (B) at (0,-2.9) {};
            \node[n_inactive] (C) at (0.3,-4.2) {};
            \node[n_inactive] (D) at (1.4, -4.5) {};
    
            \draw[->, e_active] (B) -- (A);
            \draw[->, e_active] (C) -- (A);
            \draw[->, e_active] (D) -- (A);
            \draw[<->, e_active] (C) -- (D);
        \end{tikzpicture}
        \vspace*{0.5em}
        \caption*{\footnotesize Edge Aggregation}
    \end{subfigure}%
    ~
    \begin{subfigure}[t]{0.18\textwidth}
        \centering
        \begin{tikzpicture}
            % Node Update
            \draw[draw=white] (-0.6,-3.25) rectangle (2,3.25) {};
            \draw[rounded corners] (-0.6,-1.5) rectangle (2,1.5) {};
            \node[text=tum-dark-blue] (T) at (1.65,1.2) {\scriptsize $\uu$};
            \node[text=tum-gray] (T) at (1.7,0.9) {\scriptsize $\overline{\ee}'$};
    
            \node[n_inactive, inner sep=1, circle split, circle split part fill={tum-light-orange,tum-lighter-blue}] (A) at (1,0) {\scriptsize \textcolor{tum-orange}{$\vv_i'$} \nodepart{lower} \textcolor{tum-dark-blue}{\scriptsize $\overline{\ee}_i'$}};
            \node[n_inactive] (B) at (0,0.6) {};
            \node[n_inactive] (C) at (0.3,-0.7) {};
            \node[n_inactive] (D) at (1.4, -1) {};
    
            \draw[->, e_inactive] (B) -- (A);
            \draw[->, e_inactive] (C) -- (A);
            \draw[->, e_inactive] (D) -- (A);
            \draw[<->, e_inactive] (C) -- (D);
        \end{tikzpicture}
        \vspace*{0.5em}
        \caption*{\footnotesize Node Update}
    \end{subfigure}%
    ~
    \begin{subfigure}[t]{0.18\textwidth}
        \centering
        \begin{tikzpicture}
            % Node Aggregation
            \draw[draw=white] (-0.6,-3.25) rectangle (2,3.25) {};
            \draw[rounded corners] (-0.6,-1.5) rectangle (2,1.5) {};
            \node[text=tum-gray] (T) at (1.65,1.2) {\scriptsize $\uu$};
            \node[text=tum-gray] (T) at (1.7,0.9) {\scriptsize $\overline{\ee}'$};
            \node[text=tum-orange] (T) at (1.7,0.6) {\scriptsize $\overline{\vv}'$};

            \node[n_active] (A) at (1,0) {};
            \node[n_active] (B) at (0,0.6) {};
            \node[n_active] (C) at (0.3,-0.7) {};
            \node[n_active] (D) at (1.4, -1) {};
    
            \draw[->, e_inactive] (B) -- (A);
            \draw[->, e_inactive] (C) -- (A);
            \draw[->, e_inactive] (D) -- (A);
            \draw[<->, e_inactive] (C) -- (D);
        \end{tikzpicture}
        \vspace*{0.5em}
        \caption*{\footnotesize Node Aggregation}
    \end{subfigure}%
    ~
    \begin{subfigure}[t]{0.18\textwidth}
        \centering
        \begin{tikzpicture}
            %Global Update
            \draw[draw=white] (-0.6,-3.25) rectangle (2,3.25) {};
            \draw[rounded corners] (-0.6,-1.5) rectangle (2,1.5) {};
            \node[text=tum-orange] (T) at (1.7,1.2) {\scriptsize $\uu'$};
            \node[text=tum-dark-blue] (T) at (1.7,0.9) {\scriptsize $\overline{\ee}'$};
            \node[text=tum-dark-blue] (T) at (1.7,0.6) {\scriptsize $\overline{\vv}'$};


            \node[n_inactive] (A) at (1,0) {};
            \node[n_inactive] (B) at (0,0.6) {};
            \node[n_inactive] (C) at (0.3,-0.7) {};
            \node[n_inactive] (D) at (1.4, -1) {};
    
            \draw[->, e_inactive] (B) -- (A);
            \draw[->, e_inactive] (C) -- (A);
            \draw[->, e_inactive] (D) -- (A);
            \draw[<->, e_inactive] (C) -- (D);
        \end{tikzpicture}
        \vspace*{0.5em}
        \caption*{\footnotesize Global Update}
    \end{subfigure}

    \captionsetup{width=\dimexpr\textwidth-1.5cm\relax}
    \caption{Schematic depiction of the GN block computation phases. 
    \textcolor{tum-orange}{Orange} attributes are updated or created 
    while \textcolor{tum-dark-blue}{blue} ones contribute to the corresponding 
    update.}

\end{figure}

\textcolor{red}{TODO: Describe EGN model!}