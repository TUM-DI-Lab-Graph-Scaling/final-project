\section{Preliminaries}

\subsection{Graph Neural Networks}

In most of the contemporary supervised machine learning tasks, we are usually given 
datasets
\[ 
    \mathcal{D} \: \coloneqq \: 
    \setcomp{(\xx_i, \yy_i)}{i=1, \dots, n}, 
    \quad \text{with} \quad \xx_i \in \R^{d}, \: \yy_i \in \R^k 
\]
where the inputs $\xx_i$ are assumed to be independently and identically 
distributed and we want to generally predict the labels $\yy$ from the $\xx$ using
some parametric function
\[
    \fct{f_\theta}{\R^d}{\R^k}, \quad
    \hat{\yy} \: \coloneqq \: f_\theta(\xx),
\]
i.e. we want to find some parameter $\theta$ in a parameter space $\Theta$ such that
the predictions $\hat{\yy}$ are somewhat close to the targets $\yy$ which is usually
done by minimizing a loss function (sometimes modeled as negative log-likelihood)
\[
    L(\mathcal{D}, \theta) \: \coloneqq \: \sum_{i=1}^n l(f_\theta(\xx_i), \yy_i)
\]
over the parameter space. Over the last decade, neural networks have proven to work 
really well for implementing these functions $f_\theta$ and are usually trained by 
variants of gradient descent.

However, for many complex structured non-Euclidean types of data, this approach might 
not be the best fit. For example, just consider the two following settings:
\begin{itemize}
    \item Suppose we are given molecule structures as input graphs $\xx_i$ and, as an 
          outcome, we want to predict the energy $\yy_i$ of the respective structures. 
          Clearly, there is no straightforward way to embed a graph into some $\R^d$.
    \item Suppose our inputs $\xx_i$ are certain features of users in a social network
          and we want to predict interaction with some newly uploaded content. In this 
          case, the iid assumption of the inputs $\xx_i$ is unreasonable because users
          themselves might interact with each other. Therefore it would be sensible to
          incorporate the existing connections between the data points somehow---so 
          essentially one would like models which operate on one giant input graph with 
          interconnected and dependent data points as nodes.
\end{itemize}
Surely, one could find some representation of the above input data and try to use some
conventional method (i.e. try to let a neural network learn these structures on its own),
however this would not only be inefficient but also completely unfeasible for many large 
modern datasets. 

However, for simpler structured data like sequences or images there already exist neural
network architectures that exploit these structures, like Recurrent Neural Networks (RNNs) and
Convolutional Neural Networks (CNNs) respectively. Particularly the introduction of CNNs 
in the early 2010s marked a huge step forwards for tasks like image classification 
compared to dense fully-connected neural networks: CNNs leverage the spatial structure of
images as rectangular grids by applying convolutional filters like sliding windows on small
rectangular parts of the input image. 

Even though the topological structure of sequences and images is quite simple and regular
compared to some general graph, we can still see both of them as graphs themselves. Graph Neural 
Networks (GNNs) can now be regarded as a generalization of CNNs and RNNs to general graph structures.
More precisely, this means the following:
\begin{itemize}
    \item Both RNNs and CNNs can be formulated in the language of GNNs.
    \item With GNNs one can express both of the above scenarios.
    \item GNNs exploit the structure of the input graph(s) by incorporating very
          reasonable assumptions about its predicting functions like equivariance
          (in the case of node prediction tasks) or invariance (for example in 
          the case of global predictions) with respect to permutations of the input
          nodes.
\end{itemize}

\textcolor{red}{TODO: Add detailed description of GN/EGN framework, add figures, shorten text (maybe a bit less detailed/verbose?)
}