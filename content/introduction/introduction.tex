\section{Introduction}

\subsection{Motivation}

Training large-scale models is the Formula 1 of the Machine learning research community: Large labs and corporations with 
almost unlimited resources compete to outperform each other in model size and accuracy improvements to the decimal points. 
Consequently, in recent years there has been a surge of more efficient and scalable methods to train deep learning models 
in parallel on multiple GPUs or even multiple machines. Most of the methods are desinged for large language models, however 
also offer general solutions to problems such as linear scalability, memory efficiency and 
communication reduction. We therefore adopted some of these scaling methods and applied them to machine learning models 
that operate on graphs. 

Graphs offer a powerful way to structure connected data by elevating the relationships in-between data points. 
Fundamentally a graph consists of a set of nodes, which are connected to each other by a set of edges. Both node 
and edges may contain features to further describe the entity represented by a node or the relationship between entities.
Graphs are used in a wide range of application areas such as social analysis by representing people and their relationship 
to each other as graphs or in the medical domain by modeling population graphs for disease prediction or using molecular graphs 
for drug discovery. Graphs also have a long history of applications on quantum chemistry, representing molecular 
structures and the energy as well as forces of the structure as graphs. 

In recent years, geometric deep learning, a machine learning approach to bring deep learning models 
to the graph domain, is getting more and more popular since they archieve comparable or even better 
results than traditional analytical graph algorithms often at much lower computational costs. Graph Neural 
Networks (GNNs) are currently the dominant architecture for designing geomtric deep learning models and 
are based on the message passing paradigm, where nodes gather features from neighbors, transform them via 
differentiable functions and scatter to outgoing neighboring nodes. Especially for molecular predictions, 
GNN-based architectures are orders of magnitudes faster than traditional methods from quantum chemistry. 

\subsection{Problem definition}

Molecular simulations are known to be very computationally intensive. The use of machine learning models, in particular 
Graph Neural Networks, has been very successful and lead to a shift in predicting such simulations instead of 
fully computing them. However, although ML models have been very successful, the problem remains that training, 
even on high-end hardware and multiple GPUs, takes a long time and has extremely high memory requirements. In 
addition, using hyper-optimization requires many iterations with different parameters to find the optimal model. 
Any memory and runtime savings will result in days to weeks of less training time and significantly reduced costs.

Since 2020, Microsoft is developing DeepSpeed, which is designed to reduce computing costs and memory 
requirements of large distributed models through better parallelization and minimal state replication. DeepSpeed 
runs on top of PyTorch and thus integrates into existing model training pipelines. However, DeepSpeed was developed 
for large transformer models and, to the best of our knowledge, never applied to large-scale training of Graph Neural Networks.

In our work, we forked the OCP project to integrate DeepSpeed into the training pipeline of the most widely-used 
and successful models in energy and force predicition of catalysts. 
We benchmark our approach by evaluating the epoch runtime and memory usage of the GNN architectures GemNet and 
DimeNet++ using different configurations of DeepSpeed on the OC20 dataset.


\subsection{Contributions}

Our main contributions are:

\begin{enumerate}
    \item We scaled force and energy predictions in atomic simulations of catalysts with Graph Neural Networks 
    by integrating Microsoft DeepSpeed into the Open Catalyst Project.
    \footnote{\url{https://github.com/TUM-DI-Lab-Graph-Scaling/ocp}}
    \item We evaluated the effectiveness of DeepSpeed features, in particular the Zero Redundancy Optimizer, on 
    real-world graph models in contrast to only applying it to large-scale transformer models.
    \item We developed a profiling module for the OCP codebase to allow deeper investigations of CPU/GPU memory 
    usage and model runtimes.
\end{enumerate}

\subsection{Structure}

Our final report is structured as follows: Section~\ref{section:preleminaries} follows with an overview of 
important preliminary background topics of our work. Section~\ref{section:implementation} describes important 
scaling techniques from DeepSpeed and our work to integrate them into the OCP repository. In 
section~\ref{section:evaluation}, we provide benchmark data for our changes and evaluate them in detail. 
Section~\ref{section:related_work} presents similiar literature to our work, while section~\ref{section:project_organization} 
gives a brief review of our project organization. Finally, section~\ref{section:conclusion} wraps up the 
report with a conclusion.