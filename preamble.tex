% !TeX encoding = UTF-8
% !TeX spellcheck = en_US
\documentclass[12pt, a4paper, oneside, final]{article}
\usepackage[english]{babel}
\usepackage{amsmath}
\usepackage{amssymb}
\usepackage{graphicx}
\usepackage{color}
\usepackage{xcolor}
%\usepackage{mdframed}
\usepackage[ansinew]{inputenc}
%\usepackage[round]{natbib}
\usepackage{amsmath}
\usepackage{caption}
\usepackage{subcaption}
\usepackage[]{algorithm2e}
\usepackage[ansinew]{inputenc}
\usepackage{hyperref}
\usepackage[]{algorithm2e}
\usepackage[official]{eurosym}
\usepackage{siunitx}
\usepackage{float}
\usepackage{xcolor}
\usepackage{datetime}
\usepackage{mathtools}
\usepackage{tikz}

%%%%%%%%%%%%%%%%%%%%%%%%%%%%%%%%%%%%%%%%%%%%%%%%%%%%%%%%%%%%%%%%%%%%%%%%%%%%%%%%%%%%%%%%%%
% Custom packages
%%%%%%%%%%%%%%%%%%%%%%%%%%%%%%%%%%%%%%%%%%%%%%%%%%%%%%%%%%%%%%%%%%%%%%%%%%%%%%%%%%%%%%%%%%

\usepackage{csquotes}
\usepackage{enumitem}
\usepackage{wrapfig}
\usepackage{listings}
\usepackage{xcolor}

\usetikzlibrary{shapes,snakes}
\usetikzlibrary{patterns}

%%%%%%%%%%%%%%%%%%%%%%%%%%%%%%%%%%%%%%%%%%%%%%%%%%%%%%%%%%%%%%%%%%%%%%%%%%%%%%%%%%%%%%%%%%

\newdateformat{mydate}{\shortmonthname[\the\month]  \the\year}

\setlength{\voffset}{-28.4mm}
\setlength{\hoffset}{-1in}
\setlength{\topmargin}{20mm}
\setlength{\oddsidemargin}{25mm}
\setlength{\evensidemargin}{25mm}
\setlength{\textwidth}{160mm}

%\setlength{\parindent}{0pt}

\setlength{\textheight}{235mm}
\setlength{\footskip}{20mm}
\setlength{\headsep}{50pt}
\setlength{\headheight}{0pt}

\usepackage[
    backend=biber,
    style=alphabetic,
  ]{biblatex}
  
\addbibresource{bibliography.bib}

\graphicspath{{./figures/}}

%%%%%%%%%%%%%%%%%%%%%%%%%%%%%%%%%%%%%%%%%%%%%%%%%%%%%%%%%%%%%%%%%%%%%%%%%%%%%%%%%%%%%%%%%%
% Custom definitions can be added here
%%%%%%%%%%%%%%%%%%%%%%%%%%%%%%%%%%%%%%%%%%%%%%%%%%%%%%%%%%%%%%%%%%%%%%%%%%%%%%%%%%%%%%%%%%

\newcommand{\C}{\mathbb{C}}
\newcommand{\R}{\mathbb{R}}
\newcommand{\Q}{\mathbb{Q}}
\newcommand{\Z}{\mathbb{Z}}
\newcommand{\N}{\mathbb{N}}
\newcommand{\Set}[1]{\ensuremath{\left\{\, #1 \,\right\}}}
\newcommand{\set}[1]{\ensuremath{\left\{#1\right\}}}
\newcommand{\setcomp}[2]{\Set{#1 \; \middle| \; #2}}
\newcommand{\fctmap}[5]{\ensuremath{#1 \colon #2 \to #3, \; #4 \mapsto #5 }}
\newcommand{\fct}[3]{\ensuremath{#1 \colon #2 \to #3}}
\newcommand{\abs}[1]{\left| #1 \right|}
\newcommand{\absl}[1]{\left| \, #1 \,\right|}
\newcommand{\norm}[1]{\left\lVert#1\right\rVert}
\newcommand{\norml}[1]{\left\lVert \, #1\, \right\rVert}

% bold variables
\newcommand{\xx}{\boldsymbol{x}}
\newcommand{\yy}{\boldsymbol{y}}
\newcommand{\zz}{\boldsymbol{z}}
\newcommand{\uu}{\boldsymbol{u}}
\newcommand{\vv}{\boldsymbol{v}}
\newcommand{\ee}{\boldsymbol{e}}
\newcommand{\ttt}{\boldsymbol{t}}
\newcommand{\XX}{\boldsymbol{X}}
\newcommand{\eee}{\boldsymbol{e}}
\newcommand{\hh}{\boldsymbol{h}}
\newcommand{\mm}{\boldsymbol{m}}
\newcommand{\aaa}{\boldsymbol{a}}

%%%%%%%%%%%%%%%%%%%%%%%%%%%%%%%%%%%%%%%%%%%%%%%%%%%%%%%%%%%%%%%%%%%%%%%%%%%%%%%%%%%%%%%%%%
% Official TUM colors
%%%%%%%%%%%%%%%%%%%%%%%%%%%%%%%%%%%%%%%%%%%%%%%%%%%%%%%%%%%%%%%%%%%%%%%%%%%%%%%%%%%%%%%%%%

% Main color
\definecolor{tum-blue}{RGB}{0,101,189}

% Secondary colors
\definecolor{tum-dark-blue}{RGB}{0,82,147}
\definecolor{tum-light-blue}{RGB}{100,160,200}
\definecolor{tum-lighter-blue}{RGB}{152,198,234}
\definecolor{tum-gray}{RGB}{153,153,153}

% Emphasis
\definecolor{tum-orange}{RGB}{227,114,34}
\definecolor{tum-light-orange}{RGB}{252, 214, 184}
\definecolor{tum-light-green}{RGB}{194, 240, 224}
\definecolor{tum-green}{RGB}{112, 194, 165}
\definecolor{tum-dark-green}{RGB}{30, 123, 90}
\definecolor{tum-light-gray}{RGB}{218,215,203}
\definecolor{tum-lighter-gray}{RGB}{244, 244, 240}

%%%%%%%%%%%%%%%%%%%%%%%%%%%%%%%%%%%%%%%%%%%%%%%%%%%%%%%%%%%%%%%%%%%%%%%%%%%%%%%%%%%%%%%%%%
% Python code
%%%%%%%%%%%%%%%%%%%%%%%%%%%%%%%%%%%%%%%%%%%%%%%%%%%%%%%%%%%%%%%%%%%%%%%%%%%%%%%%%%%%%%%%%%

% Credits to https://github.com/olivierverdier/python-latex-highlighting

\renewcommand*{\lstlistlistingname}{Code Listings}
\renewcommand*{\lstlistingname}{Code Listing}
\definecolor{gray}{gray}{0.5}
\colorlet{commentcolour}{tum-orange}

\colorlet{stringcolour}{tum-orange}
\colorlet{keywordcolour}{tum-dark-green}
\colorlet{exceptioncolour}{red}
\colorlet{commandcolour}{tum-dark-blue}
\colorlet{numpycolour}{blue!60!green}
\colorlet{literatecolour}{tum-dark-blue}
\colorlet{promptcolour}{green!50!black}
\colorlet{specmethodcolour}{tum-dark-blue}

\newcommand*{\framemargin}{3ex}

\newcommand*{\literatecolour}{\textcolor{literatecolour}}

\newcommand*{\pythonprompt}{\textcolor{promptcolour}{{>}{>}{>}}}

\lstdefinestyle{mypython}{
%\lstset{
%keepspaces=true,
language=python,
showtabs=true,
tab=,
tabsize=2,
basicstyle=\ttfamily\footnotesize,%\setstretch{.5},
stringstyle=\color{stringcolour},
showstringspaces=false,
alsoletter={1234567890},
otherkeywords={\%, \}, \{, \&, \|},
keywordstyle=\color{keywordcolour}\bfseries,
emph={and,break,class,continue,def,yield,del,elif ,else,%
except,exec,finally,for,from,global,if,import,in,%
lambda,not,or,pass,print,raise,return,try,while,assert,with},
emphstyle=\color{blue}\bfseries,
emph={[2]True, False, None},
emphstyle=[2]\color{keywordcolour},
emph={[3]object,type,isinstance,copy,deepcopy,zip,enumerate,reversed,list,set,len,dict,tuple,xrange,append,execfile,real,imag,reduce,str,repr},
emphstyle=[3]\color{commandcolour},
emph={Exception,NameError,IndexError,SyntaxError,TypeError,ValueError,OverflowError,ZeroDivisionError},
emphstyle=\color{exceptioncolour}\bfseries,
%upquote=true,
morecomment=[s]{"""}{"""},
commentstyle=\color{commentcolour}\slshape,
%emph={[4]1, 2, 3, 4, 5, 6, 7, 8, 9, 0},
emph={[4]ode, fsolve, sqrt, exp, sin, cos,arctan, arctan2, arccos, pi,  array, norm, solve, dot, arange, isscalar, max, sum, flatten, shape, reshape, find, any, all, abs, plot, linspace, legend, quad, polyval,polyfit, hstack, concatenate,vstack,column_stack,empty,zeros,ones,rand,vander,grid,pcolor,eig,eigs,eigvals,svd,qr,tan,det,logspace,roll,min,mean,cumsum,cumprod,diff,vectorize,lstsq,cla,eye,xlabel,ylabel,squeeze},
emphstyle=[4]\color{numpycolour},
emph={[5]__init__,__add__,__mul__,__div__,__sub__,__call__,__getitem__,__setitem__,__eq__,__ne__,__nonzero__,__rmul__,__radd__,__repr__,__str__,__get__,__truediv__,__pow__,__name__,__future__,__all__},
emphstyle=[5]\color{specmethodcolour},
emph={[6]assert,yield},
emphstyle=[6]\color{keywordcolour}\bfseries,
emph={[7]range},
emphstyle={[7]\color{keywordcolour}\bfseries},
% emph={[7]self},
% emphstyle=[7]\bfseries,
literate=*%
{:}{{\literatecolour:}}{1}%
{=}{{\literatecolour=}}{1}%
{-}{{\literatecolour-}}{1}%
{+}{{\literatecolour+}}{1}%
{*}{{\literatecolour*}}{1}%
{**}{{\literatecolour{**}}}2%
{/}{{\literatecolour/}}{1}%
{//}{{\literatecolour{//}}}2%
{!}{{\literatecolour!}}{1}%
%{(}{{\literatecolour(}}{1}%
%{)}{{\literatecolour)}}{1}%
{[}{{\literatecolour[}}{1}%
{]}{{\literatecolour]}}{1}%
{<}{{\literatecolour<}}{1}%
{>}{{\literatecolour>}}{1}%
{>>>}{\pythonprompt}{3}%
,%
%aboveskip=.5ex,
frame=trbl,
%frameround=tttt,
%framesep=.3ex,
rulecolor=\color{black!40},
%framexleftmargin=\framemargin,
%framextopmargin=.1ex,
%framexbottommargin=.1ex,
%framexrightmargin=\framemargin,
%framexleftmargin=1mm, framextopmargin=1mm, frame=shadowbox, rulesepcolor=\color{blue},#1
%frame=tb,
backgroundcolor=\color{white},
breakindent=.5\textwidth,frame=single,breaklines=true%
%}
}

\newcommand*{\inputpython}[3]{\lstinputlisting[firstline=#2,lastline=#3,firstnumber=#2,frame=single,breakindent=.5\textwidth,frame=single,breaklines=true,style=mypython]{#1}}

\lstnewenvironment{python}[1][]{\lstset{style=mypython}}{}

\lstdefinestyle{mypythoninline}{
style=mypython,%
basicstyle=\ttfamily,%
keywordstyle=\color{keywordcolour},%
emphstyle={[7]\color{keywordcolour}},%
emphstyle=\color{exceptioncolour},%
literate=*%
{:}{{\literatecolour:}}{2}%
{=}{{\literatecolour=}}{2}%
{-}{{\literatecolour-}}{2}%
{+}{{\literatecolour+}}{2}%
{*}{{\literatecolour*}}2%
{**}{{\literatecolour{**}}}3%
{/}{{\literatecolour/}}{2}%
{//}{{\literatecolour{//}}}{2}%
{!}{{\literatecolour!}}{2}%
%{(}{{\literatecolour(}}{2}%
%{)}{{\literatecolour)}}{2}%
{[}{{\literatecolour[}}{2}%
{]}{{\literatecolour]}}{2}%
{<}{{\literatecolour<}}{2}%
{<=}{{\literatecolour{<=}}}3%
{>}{{\literatecolour>}}{2}%
{>=}{{\literatecolour{>=}}}3%
{==}{{\literatecolour{==}}}3%
{!=}{{\literatecolour{!=}}}3%
{+=}{{\literatecolour{+=}}}3%
{-=}{{\literatecolour{-=}}}3%
{*=}{{\literatecolour{*=}}}3%
{/=}{{\literatecolour{/=}}}3%
%% emphstyle=\color{blue},%
}

\newcommand*{\pyth}{\lstinline[style=mypythoninline]}


%%%%%%%%%%%%%%%%%%%%%%%%%%%%%%%%%%%%%%%%%%%%%%%%%%%%%%%%%%%%%%%%%%%%%%%%%%%%%%%%%%%%%%%%%%
% JSON syntax highlighting
%%%%%%%%%%%%%%%%%%%%%%%%%%%%%%%%%%%%%%%%%%%%%%%%%%%%%%%%%%%%%%%%%%%%%%%%%%%%%%%%%%%%%%%%%%


\colorlet{punct}{red!60!black}
\definecolor{delim}{RGB}{20,105,176}
\colorlet{numb}{magenta!60!black}
\definecolor{jsonStringColor}{RGB}{6,125,23}

\lstdefinestyle{myjson}{
    basicstyle=\ttfamily\footnotesize,%\setstretch{.5},
    showstringspaces=false,
    breaklines=true,
    stringstyle=\color{jsonStringColor},
    string=[s]{"}{"},
    literate=
     *{0}{{{\color{numb}0}}}{1}
      {1}{{{\color{numb}1}}}{1}
      {2}{{{\color{numb}2}}}{1}
      {3}{{{\color{numb}3}}}{1}
      {4}{{{\color{numb}4}}}{1}
      {5}{{{\color{numb}5}}}{1}
      {6}{{{\color{numb}6}}}{1}
      {7}{{{\color{numb}7}}}{1}
      {8}{{{\color{numb}8}}}{1}
      {9}{{{\color{numb}9}}}{1}
      {:}{{{\color{punct}{:}}}}{1}
      {,}{{{\color{punct}{,}}}}{1}
      {\{}{{{\color{delim}{\{}}}}{1}
      {\}}{{{\color{delim}{\}}}}}{1}
      {[}{{{\color{delim}{[}}}}{1}
      {]}{{{\color{delim}{]}}}}{1},
    ,%
    %aboveskip=.5ex,
    frame=trbl,
    %frameround=tttt,
    %framesep=.3ex,
    rulecolor=\color{black!40},
    %framexleftmargin=\framemargin,
    %framextopmargin=.1ex,
    %framexbottommargin=.1ex,
    %framexrightmargin=\framemargin,
    %framexleftmargin=1mm, framextopmargin=1mm, frame=shadowbox, rulesepcolor=\color{blue},#1
    %frame=tb,
    backgroundcolor=\color{white},
    breakindent=.5\textwidth,frame=single,breaklines=true%
    %}
}

\newcommand*{\inputjson}[3]{\lstinputlisting[firstline=#2,lastline=#3,firstnumber=#2,frame=single,breakindent=.5\textwidth,frame=single,breaklines=true,style=myjson]{#1}}

\lstnewenvironment{json}[1][]{\lstset{style=myjson}}{}


    